\input opmac

% Increase the gap between paragraphs.
\parskip=1\baselineskip

% Define a special font fo sub title.
\font\tensf=cmss10
\let\subtitlefont=\tensf

\def\subtitle#1{\noindent {\subtitlefont #1}\par}

% Define the conv binary op
\def\conv{\otimes}

% Call the indentation for descriptions \descindent and set it to 6 picas.
\newdimen\descindent \descindent = 6pc
\def\itembox#1{\noindent\llap{\hbox to \descindent{\bf #1\hfil}}}

%%%%%

\subtitle{Conv2D and its Gradients from Math Perspective}

\noindent Conv2D is quite simple. Given $X, W$, and $Y$, where $X$ and $Y$ are
the input and output respectively and $W$ is a typically quite small kernel, we
denote the formula as
%
$$ Y = X \conv W. $$

\noindent Assuming the padding is {\tt same}, which means the shape of the
output $Y$ is same as the shape of input $X$, denocated as $<m, n>$ and, $W$ has
odd number of rows and columns, denoted as $<a,b>$, then we have
%
$$ Y_{i,j} =
    \sum_{x=0}^{a-1} \sum_{y=0}^{b-1}
        X_{i+x-{a-1 \over 2},j+y-{b-1 \over 2}} W_{x,y}.
$$


\bye
