\input opmac
\parskip=1\baselineskip

\noindent Tensor a multi-dimension float-value array. For a Tensor with shape $<a, b, c>$,
where $a,b,$ and $c$ are all positive integers, there are in total of $ a b c $
number of elements, called {\tt size}, in it. The length of the shape is called
{\tt rank}. It is also the number of dimensions. Scalar is viewed as a Tensor
with shape $<1>$, i.e., rank $1$.

\noindent This leads to the following definition:

\begtt
typedef struct {
  mlvm_uint_t  rank;   /* Must be positive (non-zero). */
  mlvm_uint_t* shape;  /* Length is `rank` above. */
  mlvm_size_t  size;   /* Total number of elements. */
  double*      value;  /* The value buffer. */
} tensor_t;
\endtt

\noindent Implementation wise there are two new concepts.

{\narrower
\noindent{\bf value mode}\quad The first one is ownership, called {\tt value-mode\_}, which
is an internal field. It represents whether the current Tensor owns its value
buffer.

\noindent{\bf stride}\quad The second one is {\tt stride}, with type {\tt mlvm\_size\_t*}.
It is an array with same length as {\tt shape}. For {\tt stride[i]}, it
represents the offset difference between the dimension increment at dimension
{\tt shape[i]}.
\par}

\bye
