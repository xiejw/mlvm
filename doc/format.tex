% Defines the section. Use a special font and avoid the indent for the following
% paragraph.
\font\secfont=cmss10
\def\section #1{
  \vskip\baselineskip
  \noindent{\secfont{}#1}
  \hfill\vskip\baselineskip
  \everypar={{\setbox0\lastbox}\everypar={}}}

% Defines a \symbol macro for coding related nouns
%
% * It starts with a group and redfines the undercore (_) to be normal char
% * As all arguments are read when process macro. The macro should be defined as
%   two macros, so the underscore in the argument can be handled correctly.
% * \relax makes the \catcode effective immediately. A {} (empty group) or empty
%   space can work also. This is the space-after-number rule (See TexBook Page
%   208).
\font\ninett=cmtt9
\def\symbol{\begingroup\catcode`\_=12\relax\symbolimpl}
\def\symbolimpl#1{{\ninett #1}\endgroup}

% Defines a common page size for notes.
\def\notespagesize{\hsize=5in{}\vsize=7in}
